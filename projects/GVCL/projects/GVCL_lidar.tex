\documentclass{article}
\usepackage{fullpage}
\usepackage{lmodern, hyperref,graphicx}
\usepackage[T1]{fontenc}
\usepackage{lmodern,graphicx,amsmath}
\usepackage[noadjust]{cite}

\hypersetup{colorlinks=true}

\newcommand{\bdesc}{\begin{description}}
\newcommand{\edesc}{\end{description}}
\newcommand{\beqna}{\begin{eqnarray}}
\newcommand{\eeqna}{\end{eqnarray}}
\newcommand{\benum}{\begin{enumerate}}
\newcommand{\eenum}{\end{enumerate}}
\newcommand{\bitem}{\begin{itemize}}
\newcommand{\eitem}{\end{itemize}}
\newcommand{\commenttxt}[1]{}
\newcommand{\subheading}[1]{\vspace{1cm}\noindent{\bf #1}}

\title{Three-dimensional Visualization of LiDAR Point Cloud Data Using
  Structural Feature Extraction}
\author{Graphics-Visualization-Computing Lab \\
International Institute of Information Technology, Bangalore}
\date{September 2013}
\begin{document}
\maketitle
The LiDAR (Light Detection and Ranging) technology is used
heavily for exploration of topographic data, which is procured through high
speed airborne altimetry. Our motivation is to extract features of interest
from processed LiDAR data given in .las files and use these features to
reduce the point cloud. Our reduction is based on structural and geometric
information which has been extracted using topological information of the
data. Our work closely follows the work done by Keller et al.[1].

\section*{References}
[1] P. Keller, O. Kreylos, M. Vanco, M. Hering-Bertram, E. S.
Cowgill, L. H. Kellogg, B. Hamann, and H. Hagen, ``Extracting
and Visualizing Structural Features in Environmental Point
Cloud LiDaR Data Sets.'' Heidelberg, Germany: Springer-
Verlag, 2010, pp. 179--193.

\end{document}
