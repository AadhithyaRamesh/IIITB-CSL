\documentclass{article}
\usepackage{fullpage}
\usepackage{lmodern, hyperref,graphicx}
\usepackage[T1]{fontenc}
\usepackage{lmodern,graphicx,amsmath}
\usepackage[noadjust]{cite}

\hypersetup{colorlinks=true}

\newcommand{\bdesc}{\begin{description}}
\newcommand{\edesc}{\end{description}}
\newcommand{\beqna}{\begin{eqnarray}}
\newcommand{\eeqna}{\end{eqnarray}}
\newcommand{\benum}{\begin{enumerate}}
\newcommand{\eenum}{\end{enumerate}}
\newcommand{\bitem}{\begin{itemize}}
\newcommand{\eitem}{\end{itemize}}
\newcommand{\commenttxt}[1]{}
\newcommand{\subheading}[1]{\vspace{1cm}\noindent{\bf #1}}

\title{Three-dimensional Interactive Visualization of ARGO Data}
\author{Graphics-Visualization-Computing Lab \\
International Institute of Information Technology, Bangalore}
\date{July 2014}
\begin{document}
\maketitle

ARGO floats are deployed in the oceans to measure temperature-salinity profiles
at concerned points. These measurements include several other relevant
dependent variables, such as, chemical concentrations, current profile, etc. In
terms of visualization, it becomes a multi-variate scattered point data. Our
motivation to find effective visualizations of time-varying ARGO float data,
and study spatio-temporal patterns of the same. We are looking towards
integrating UCAR's VAPOR [1] visualization for our needs.

\section*{References}
[1] VAPOR software, \href{https://www.vapor.ucar.edu/}{https://www.vapor.ucar.edu/}
\end{document}
